%_____________________________________________________________________________________________ 
% LATEX Template: Department of Comp/IT BTech Project Reports
% Sample Chapter
% Sun Mar 27 10:25:35 IST 2011
%
% Note: Itemization, enumeration and other things not shown. A sample figure is included.
%_____________________________________________________________________________________________ 
% \setstretch{1.5}
\chapter{Introduction}

The positioning or the layout of the centerline of the route or road on the ground is called route alignment. Aligning routes requires consideration of a variety of factors. These various factors that need to be considered for route alignment include, but are not limited to, the type of vehicle traffic, gradient, climate, topography, obstructions, economic constraints, social implications of the alignment plan, reduction of harm caused to the environment by the plan etc. Route alignment is imperative because incorrect alignment of a route can increase the number of accidents, road construction costs, vehicles operation cost, and road maintenance costs. It can also cause discomfort to road users, and reduce the road's durability. The alignment of a new route should be carefully considered as it affects the cost of construction, maintenance, safety and ease of travel. In aligning routes, the following aspects should also be taken into consideration: 
\begin{enumerate}
    \item The optimal route, the construction and maintenance should be as low as possible.
    \item The operational costs should be as low as possible.
    \item The maximum degree of comfort and safety must be maintained.
    \item Aesthetic considerations.
\end{enumerate}


Because of the consideration and analysis of various datasets, selecting the best route alignment is a complex process.It is possible to model these datasets easily using GIS . Thus, In order to facilitate route alignment, GIS and Remote Sensing can play a very crucial role when used together. Remote sensing began in the 1840s. In remote sensing, information is obtained from a distance, usually from satellites, about different areas and entities. The field of geographic information systems (GIS) started in the 1960s and is a database that contains geographic data and software tools for organizing, analyzing, and displaying that data. As a result of the use of GIS and remote sensing, route alignment is less time-consuming, less costly, and requires less manpower. 


Considering that GIS is a collection of geospatial data, deep learning algorithms can be applied to this data in order to construct models. Deep learning is critical for highway alignment because it consists of models, algorithms and techniques that can simplify and accelerate the process. A deep learning algorithm was first implemented in the 1960s, and since then it has been used to power a variety of applications around the world. It is widely recognized that deep learning has numerous applications, including speech recognition, image recognition, recommendation systems, natural language processing, image reconstruction among others. Essentially, deep learning is a subset of machine learning, and both fall under the category of artificial intelligence. Recent years have seen an explosion of applications of Machine Learning algorithms in GIS and Remote Sensing. GIS and remote sensing applications can use a variety of supervised and non-parametric machine learning models. Machine learning algorithms like Naïve Bayes, Support Vector Machine,Random Forest, and Decision Trees have found practical applications in GIS and remote sensing. Thus, the availability of machine learning and deep learning algorithms makes the deployment of Artificial Intelligence for route alignment possible. 


Several Artificial Intelligence concepts and algorithms are being used for route alignment purposes, such as Genetic Algorithm, Fuzzy Logic, and Swarm Intelligence. As part of the larger class of evolutionary algorithms, genetic algorithms are adaptive heuristic search algorithms. Natural selection and genetics are the foundations of genetic algorithms . The concept of swarm intelligence comprises the use of the collective knowledge of a number of objects (people, insects, etc.) for the purpose of finding the optimal solution to a specific problem.
As a form of many-valued logic, fuzzy logic allows the truth value of variables to be any real number between 0 and 1.

The main objectives of this report is to:
\begin{enumerate}
    \item Describe the methods and techniques used for aligning routes in a comprehensive manner. 
    \item Identify the applications of Machine Learning and Deep Learning in the field of GIS and Remote Sensing for the purpose of route alignment.
    \item Highlight how Artificial Intelligence is being used to make the process of route alignment faster and easier.
    \item Identify the Machine Learning, Deep Learning and AI algorithms being used for the process of route alignment.  
\end{enumerate}
%_____________________________________________________________________________________________ 
